\section{Konfidenzintervalle}
\begin{mainbox}{Definition Konfidenzintervall}
	Sei \(\alpha \in [0,1]\). Ein Konfidenzintervall für \(\theta\) mit Niveau \(1 - \alpha\) ist ein Zufallsintervall \(I=[A,B]\), sodass gilt
	\[\forall \theta \in \Theta \quad \P_\theta[A\le \theta \le B] \ge 1- \alpha\]

	wobei \(A\) und \(B\) Zufallsvariablen der Form \(A = a(X_1, \ldots, X_n), B = b(X_1, \ldots, X_n)\) mit \(a,b: \R^n \to \R\) sind.
\end{mainbox}

Wenn wir einen Schätzer \(T = T_{ML} \sim \mathcal{N}(m, \frac{1}{n})\) haben, suchen wir ein Konfidenzintervall der Form
\[I = [T-c/\sqrt{n}, T+c/\sqrt{n}]\]
Hierbei gilt:
\begin{align*}
	\P_\theta[T-c/\sqrt{n} \le m \le T+c/\sqrt{n}] \\
	= \P_\theta[-c\le Z \le c]
\end{align*}
wobei \(Z = \sqrt{n}(T-m)\) ist.

\subsection{Häufige Fälle}
\subsubsection*{\texorpdfstring{Normalverteilt - \(\mu\) unbekannt, \(\sigma^2\) bekannt (z-Test)}{Normalverteilt - μ unbekannt, σ² bekannt (z-Test)}}
Erwartungstreuer Schätzer: \(\overline{X}_n = \frac{1}{n} \sum_{i=1}^n X_i\)\\
Verteilung unter \(\P_\theta: \frac{\overline{X}_n - \theta_0}{\sqrt{\sigma^2/n}} \sim \mathcal{N}(0,1)\)
\begin{enumerate}
	\item Modell \(X_1, \ldots, X_n \sim \mathcal{N}(\theta, \sigma^2)\) uiv. unter \(\P_\theta\)
	\item Hypothesen \(H_0 : \theta = \theta_0\), z.B. \(H_A : \theta \ne \theta_0\)
	\item Test \(T = \frac{\overline{X}_n - \mu}{\sqrt{\sigma^2/n}} \sim \mathcal{N}(0,1)\)
	\item Verwerfungsbereich \(]-\infty, -c[ \ \cup \ ] c, \infty\) für \(c\ge 0\)
\end{enumerate}

\subsubsection*{\texorpdfstring{Normalverteilt - \(\mu\), \(\sigma^2\) unbekannt (t-Test)}{Normalverteilt - μ, σ² unbekannt (t-Test)}}
Wir definieren \(\vec{\theta} = (\mu, \sigma^2)\) und den Varianz-Schätzer \(S^2 = \frac{1}{n-1}\sum_{i=1}^n (X_i - \overline{X}_n)^2\).
\begin{enumerate}
	\item Modell \(X_1, \ldots, X_n \sim \mathcal{N}(\theta, \sigma^2)\) uiv. unter \(\P_{\vec{\theta}}\)
	\item Test \(T = \frac{\overline{X}_n - \mu_0}{\sqrt{S^2/n}} \sim t_{n-1}\)
\end{enumerate}

\subsection{Approximatives Konfidenzintervall}
Wir können den zentralen Grenzwertsatz benutzen, um eine standardnormalverteilte ZV zu erhalten, und damit die Konfidenzintervalle zu bestimmen.

